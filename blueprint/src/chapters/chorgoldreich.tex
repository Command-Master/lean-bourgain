\chapter{Lemmas about the Inner Product Extractor}
\label{chapr:cr}

\begin{proposition}
    \label{AddChar.eq_iff}
    \lean{AddChar.eq_iff}
    \leanok
    For a character $\chi$, $\chi(a) = \chi(b)$ iff $\chi(a-b) = 1$.
\end{proposition}


\begin{proposition}
    \label{IP_comm}
    \lean{IP_comm}
    \leanok
    The inner product is commutitive.
\end{proposition}

\begin{lemma}
    \label{apply_inner_product_injective}
    \lean{apply_inner_product_injective}
    \leanok
    If $\chi$ is a non-trivial character of a field $\mathbb{F}$, then there is an injective
    function from elements of $\mathbb{F}^2$ ({\color{red} generalize this to any dimension}) to characters of it, 
    defined by $f(x)(y) = \chi(x \cdot y)$.
\end{lemma}

\begin{proof}
    \leanok
    \uses{AddChar.eq_iff}
    It's easy to see this maps values to additive characters.
    For injectivity, we have some value $x$ such that $\chi(x) \neq 1$.
    Now if $f((a_1, a_2)) = f((b_1, b_2))$, if they aren't equal, we can apply either
    $\frac{x}{a_1 - b_1}$ or $\frac{x}{a_2 - b_2}$, and then we get $\chi(x) = 1$ by \ref{AddChar.eq_iff},
    a contradiction.
\end{proof}

\begin{lemma}
    \label{apply_inner_product_bijective}
    \lean{apply_inner_product_bijective}
    \leanok
    The function in the previous lemma is actually a bijection.
\end{lemma}

\begin{proof}
    \leanok
    \uses{apply_inner_product_injective}
    By \ref{apply_inner_product_injective} and the cardinality being equal.
\end{proof}

\begin{theorem}
    \label{bourgain_extractor_aux_inner}
    \lean{bourgain_extractor_aux_inner}
    \leanok
    \uses{apply_inner_product_bijective}

    \textbf{Note: the inner product and DFT here aren't normalized.}
    $$\sum_x{a(x)\sum_y{b(y) \chi(x \cdot y)}} = \langle a, P(\hat b) \rangle$$
    where $P$ reorders $\hat b$ based on \ref{apply_inner_product_bijective}
\end{theorem}

\begin{proof}
    \leanok
    TODO
\end{proof}

\begin{theorem}
    \label{bourgain_extractor_aux₀}
    \lean{bourgain_extractor_aux₀}
    \leanok

    $$|\sum_x{a(x)\sum_y{b(y) \chi(x \cdot y)}}|^2 \leq |A|^2 \lVert a \rVert_{\ell^2}^2 \lVert b \rVert_{\ell^2}^2$$
\end{theorem}

\begin{proof}
    \leanok
    \uses{bourgain_extractor_aux_inner}
    We use \ref{bourgain_extractor_aux_inner} to rewrite the sum, and then use Cauchy-Schwartz.
    Then we can undo the reordering and use Parseval's theorem to get the desired result.
\end{proof}

\begin{theorem}
    \label{bourgain_extractor_aux₀t}
    \lean{bourgain_extractor_aux₀'}
    \leanok

    $$|\sum_x{a(x)\sum_y{b(y) \chi(x \cdot y)}}| \leq |A| \lVert a \rVert_{\ell^2} \lVert b \rVert_{\ell^2}$$
\end{theorem}

\begin{proof}
    \leanok
    \uses{bourgain_extractor_aux₀}
    Simplying apply a square root to \ref{bourgain_extractor_aux₀}.
\end{proof}

\begin{theorem}
    \label{bourgain_extractor_aux₁}
    \lean{bourgain_extractor_aux₁}
    \leanok
    For any bilinear form $F$ and character $\chi$,
    $$|\sum_x{a(x) \sum_y{b(y) \chi(F(x, y))}}|^2 \leq |\sum_x{a(x) \sum_y{(b-b)(y) \chi(F(x, y))}}|$$
\end{theorem}

\begin{proof}
    \leanok
    \begin{align}
        |\sum_x{a(x) \sum_y{b(y) \chi(F(x, y))}}|^2 &\leq& (\sum_x{a(x) |\sum_y{b(y) \chi(F(x, y))} | })^2 \\
        &\leq& \sum_x{a(x) |\sum_y{b(y) \chi(F(x, y))} |^2}\\
        &=&\sum_x{a(x) (\sum_y{b(y) \chi(F(x, y))}) \bar{(\sum_y{b(y) \chi(F(x, y))})}} \\
        &=&\sum_x{a(x) \sum_y \sum_{y'}{b(y) b(y') \chi(F(x, y)) \chi(-F(x, y'))} } \\
        &=&\sum_x{a(x) \sum_y \sum_{y'}{b(y) b(y') \chi(F(x, y - y'))} } \\
        &=&\sum_x{a(x) \sum_y{(b-b)(y) \chi(F(x, y))} } \\
    \end{align}
\end{proof}

\begin{theorem}
    \label{bourgain_extractor_aux₁t}
    \lean{bourgain_extractor_aux₁'}
    \leanok
    For any bilinear form $F$ and character $\chi$,
    $$|\sum_x{a(x) \sum_y{b(y) \chi(F(x, y))}}| \leq \sqrt{|\sum_x{a(x) \sum_y{(b-b)(y) \chi(F(x, y))}}|}$$
\end{theorem}

\begin{proof}
    \leanok
    \uses{bourgain_extractor_aux₁}
    Trivial from \ref{bourgain_extractor_aux₁}.
\end{proof}

\begin{theorem}
    \label{bourgain_extractor_aux₂}
    \lean{bourgain_extractor_aux₂}
    \leanok
    If $a$ and $b$ are $\varepsilon$-close to $N$ entropy,
    then 
    $$|\sum_x{a(x) \sum_y{b(y) \chi(F(x, y))}}| \leq \frac{|A|}N + 2\varepsilon$$
\end{theorem}

\begin{proof}
    \leanok
    \uses{bourgain_extractor_aux₀t, filter_neg_le_inv_card_le}
    From the hypothesis and \ref{filter_neg_le_inv_card_le} we can look at $a'(x) = \begin{cases}a(x) & a(x) \leq \frac1N \\ 0 & \frac1N < a(x)\end{cases}$, and similarly
    for $b'$, and the difference would be at most $2 \varepsilon$. Then we can apply \ref{bourgain_extractor_aux₀t} to get the result.
\end{proof}

