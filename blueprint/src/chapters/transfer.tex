\chapter{Transfer operator}
\label{chap:transfer}

\begin{definition}
    \label{transfer}
    \lean{transfer}
    \leanok
    For $f : A \to B, G : A \to C$ we have $f \# g$ is a function $B \to C$ defined by
    $f \# g(x) = \sum_{f(y) = x}g(y)$.
\end{definition}

\begin{proposition}
    \label{transfer_add}
    \lean{transfer_add}
    \leanok
    \uses{transfer}
    We have $f \# (g + h) = f \# g + f \# h$.
\end{proposition}

\begin{proposition}
    \label{transfer_sub}
    \lean{transfer_sub}
    \leanok
    \uses{transfer}
    We have $f \# (g - h) = f \# g - f \# h$.
\end{proposition}

\begin{proposition}
    \label{comp_transfer}
    \lean{comp_transfer}
    \leanok
    \uses{transfer}
    If $h$ is an additive homomorphism we have $h \circ (f \# g) = f \# (g \circ h)$.
\end{proposition}

\begin{proposition}
    \label{equiv_transfer}
    \lean{equiv_transfer}
    \leanok
    \uses{transfer}
    If $f$ is a bijection we have $(f \# g)(x) = g (f^{-1}(x))$.
\end{proposition}

\begin{lemma}
    \label{transfer_transfer}
    \lean{transfer_transfer}
    \leanok
    \uses{transfer}
    We have $h \# (f \# g) = (h \circ f) \# g$.
\end{lemma}

\begin{proof}
    \leanok
    $$\sum_{y \in h^{-1}(x)} \sum_{z \in f^{-1}(y)} g(z) = \sum_{z} \sum_{y \in h^{-1}(x), z \in f^{-1}(y)} g(z) = 
    \sum_{z} [h(f(z)) = x] g(z) = \sum_{z \in (h \circ f)^{-1}(x)} g(z) = ((h \circ f) \# g) (x)$$
\end{proof}

\begin{proposition}
    \label{transfer_id}
    \lean{transfer_id}
    \leanok
    \uses{transfer}
    $\mathrm{id} \# f = f$
\end{proposition}

\begin{proposition}
    \label{transfer_sum}
    \lean{transfer_sum}
    \leanok
    \uses{transfer}
    $$\sum_x {(f \# g) (x) h(x)} = \sum_x {g(x) h(f(x))}$$
\end{proposition}

\begin{lemma}
    \label{transfer_expect}
    \lean{transfer_expect}
    \leanok
    \uses{transfer}
    $$E[(f \# g) (x) h(x)] = \frac{|A|}{|B|} E[g(x) h(f(x))]$$
\end{lemma}

\begin{proof}
    \leanok
    \uses{transfer_sum}
    By unfolding the expectation and using \ref{transfer_sum}.
\end{proof}


\begin{proposition}
    \label{transfer_ne_zero}
    \lean{transfer_ne_zero}
    \leanok
    \uses{transfer}
    if $(f \# g) (x) \neq 0$ then $\exists y, f(y) = x$.
\end{proposition}