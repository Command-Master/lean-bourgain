\chapter{Finite Probability Distributions}
\label{chap:finpmf}

\begin{definition}
    \label{FinPMF}
    \lean{FinPMF}
    \leanok
    A finite probability distribution is a function $f : A \to \mathbb{R}$ from a finite type $A$, such that $f$ is nonnegative and the sum of $f$ is
    1.
\end{definition}

\begin{definition}
    \label{Uniform}
    \lean{Uniform}
    \leanok
    \uses{FinPMF}
    The uniform distribution on a nonempty set $A$ assigns $\frac1{|A|}$ to all values in $A$ and $0$ to other values.
\end{definition}

\begin{definition}
    \label{instMulFinPMF}
    \lean{instMulFinPMF}
    \leanok
    \uses{FinPMF}
    Given two finite probability distributions $f: A \to \mathbb{R}, g : B \to \mathbb{R}$, we have a probability distribution
    from $A \times B$ defines as $(f \times g)(x, y) = f(x) g(y)$.
\end{definition}

\begin{definition}
    \label{FinPMF.apply}
    \lean{FinPMF.apply}
    \leanok
    \uses{FinPMF, transfer}
    Given a finite probability distribution $f: A \to \mathbb{R}$ and a function $g : A \to B$, we can apply $g$ to the random variable
    represented by $f$. This gives the distribution $g \# f$.
\end{definition}

We can directly transfer all theorems on $f \# g$ to finite PMFs.

\begin{definition}
    \label{instSubFinPMF}
    \lean{instSubFinPMF}
    \leanok
    \uses{FinPMF, FinPMF.apply, instMulFinPMF}
    Given two finite probability distributions $f: A \to \mathbb{R}, g : A \to \mathbb{R}$, we have a probability distribution
    defines as $f-g = s \# (f \times g)$ with $s(x, y) = x-y$.
\end{definition}

\begin{definition}
    \label{instAddFinPMF}
    \lean{instAddFinPMF}
    \leanok
    \uses{FinPMF, FinPMF.apply, instMulFinPMF}
    Given two finite probability distributions $f: A \to \mathbb{R}, g : A \to \mathbb{R}$, we have a probability distribution
    defines as $f+g = a \# (f \times g)$ with $a(x, y) = x+y$.
\end{definition}

\begin{definition}
    \label{instNegFinPMF}
    \lean{instNegFinPMF}
    \leanok
    \uses{FinPMF, FinPMF.apply, instMulFinPMF}
    Given a finite probability distribution $f: A \to \mathbb{R}$, we have a probability distribution
    defines as $-f = n \# f$ with $n(x) = -x$.
\end{definition}

\begin{proposition}
    \label{FinPMFCommMonoid}
    \lean{FinPMFCommMonoid}
    \leanok
    \uses{instAddFinPMF, instSubFinPMF, instNegFinPMF}
    These operations define a commutative monoid.
\end{proposition}


\begin{lemma}
    \label{FinPMF.apply_mul}
    \lean{FinPMF.apply_mul}
    \leanok
    \uses{FinPMF, FinPMF.apply, instMulFinPMF}
    We have $(f \# a) \times (g \# b) = h \# (a \times b)$, with $h(x, y) = (f(x), g(y))$.
\end{lemma}

\begin{proof}
    \leanok
    By calculation.
\end{proof}

\begin{lemma}
    \label{FinPMF.apply_add}
    \lean{FinPMF.apply_add}
    \leanok
    \uses{FinPMF, FinPMF.apply, instMulFinPMF, instAddFinPMF}
    We have $(f \# a) + (g \# b) = h \# (a \times b)$, with $h(x, y) = f(x) + g(y)$.
\end{lemma}

\begin{proof}
    \leanok
    \uses{FinPMF.apply_mul}
    By simplification after \ref{FinPMF.apply_mul}.
\end{proof}

\begin{definition}
    \label{FinPMF.linear_combination}
    \lean{FinPMF.linear_combination}
    \leanok
    \uses{FinPMF}
    Given a finite probability distribution $f: A \to \mathbb{R}$ and a list of finite probability distributions on $B$, indexed by
    elements of $A$, $g$, we can define $g(f)$ as the probability distribution obtained by sampling an element from $f$, and then
    sampling an elemente from the corresponding distribution in $g$.
\end{definition}


\begin{lemma}
    \label{linear_combination_linear_combination}
    \lean{linear_combination_linear_combination}
    \leanok
    \uses{FinPMF, FinPMF.linear_combination}
    We have $f(g(a)) = h(a)$ with $h(x) = g(f(x))$.
\end{lemma}

\begin{proof}
    \leanok
    By calculation.
\end{proof}

\begin{lemma}
    \label{linear_combination_apply}
    \lean{linear_combination_apply}
    \leanok
    \uses{FinPMF, FinPMF.linear_combination, FinPMF.apply}
    We have $g \# f(a) = h(a)$ with $h(x) = g\# f(x)$.
\end{lemma}

\begin{proof}
    \leanok
    By calculation.
\end{proof}

