\chapter{Lines}
\label{chap:line}

TOOD: Figure out how to write blueprints about definitions

\begin{definition}
    \label{Line}
    \lean{Line}
    \leanok
    A line over a finite ({\color{red} TODO: remove finiteness from this file}) field $\mathbb{F}$ is a
    linear subspace of $\mathbb{F}^3$ of dimension 2.
\end{definition}

\begin{definition}
    \label{Line_mem}
    \lean{mem2}
    \leanok
    \uses{Line}
    A value $(x, y) \in \mathbb{F}^2$ is in a line $L$ iff $(x, y, 1) \in L$.
\end{definition}

\begin{definition}
    \label{line_apply}
    \lean{Line.apply}
    \leanok
    \uses{Line}
    Given a linear isomorphism $P$ and a line $L$, we have a line $PL$.
\end{definition}

\begin{proof}
    \leanok
    This is a valid line because linear isomorphism preserves dimension.
\end{proof}

\begin{lemma}
    \label{apply_injective}
    \lean{apply_injective}
    \leanok
    \uses{Line, line_apply}
    For any linear equivalence, applying it to lines is injective.
\end{lemma}

\begin{proof}
    \leanok
    From the injectivity of linear isomorphisms.
\end{proof}

\begin{theorem}
    \label{CS_UB}
    \lean{CS_UB}
    \leanok
    Given a set $P$ of points and a set $L$ of lines, the number of incidences is at most
    $\sqrt{|L| |P| (|P| + |L|)}$.
\end{theorem}

TODO