\chapter{Incidence}
\label{chap:inc}

\DeclareMathOperator{\stpfone}{\varepsilon}
\DeclareMathOperator{\stpftwo}{\varepsilon_2}
\DeclareMathOperator{\sgone}{\varepsilon'}

\begin{definition}
    \label{ST_C}
    \lean{ST_C}
    \leanok
    \uses{ST_C2}
    $C = C_2 + 1$
\end{definition}

\begin{definition}
    \lean{ST_prime_field_eps}
    \leanok
    $\stpfone(\beta) = \stpftwo(\beta) / 3$
\end{definition}

\begin{theorem}
    \label{ST_prime_field}
    \lean{ST_prime_field}
    \leanok
    Let there be a set $P$ of points and a set $L$ of lines over a prime field, 
    with $|P| \leq n, |L| \leq n$ and $p^\beta \leq n \leq p^{2 - \beta}$.
    Then the number of intersections is at most $ C n^{\frac32 - \stpfone(\beta)} $.
\end{theorem}

\begin{proof}
    \leanok
    \uses{ST_prime_field_aux₂'}
    We reduce this to \ref{ST_prime_field_aux₂'}, by removing all points contained in at most
    $n^{\frac12 - \stpfone(\beta)}$ lines. This removes at most $n^{\frac32 - \stpfone(\beta)}$
    points, which is corrected for with $C = C_2 + 1$.
\end{proof}

\begin{definition}
    \label{ST_C2}
    \lean{ST_C₂}
    \leanok
    $C_2 = \sqrt{2(C_3 + \frac{\sqrt2}4)}$
\end{definition}

\begin{theorem}
    \label{ST_prime_field_aux₂'}
    \lean{ST_prime_field_aux₂'}
    \leanok
    Let there be a set $P$ of points and a set $L$ of lines over a prime field, 
    with $|P| \leq n, |L| \leq n$ and $p^\beta \leq n \leq p^{2 - \beta}$,
    and each point intersecting with at least $n^{\frac12 - \stpfone(\beta)}$ lines.
    Then the number of intersections is at most $C_2 n^{\frac32 - \stpfone(\beta)}$.
\end{theorem}

\begin{proof}
    \leanok
    \uses{ST_prime_field_aux, CS_UB}
    We reduce this to \ref{ST_prime_field_aux}, by removing all points contained in more than 
    $4 n^{\frac12 + \stpfone(\beta)}$ lines. There can be at most
    $n^{1 - 2 \stpfone(\beta)} \frac{\sqrt2}4$ such points, by \ref{CS_UB}.
    Therefore, there are still many remaining points, and because each point has at least
    $n^{\frac12 - \stpfone(\beta)}$ lines there are still many intersections.
\end{proof}

\begin{theorem}
    \label{ST_prime_field_aux}
    \lean{ST_prime_field_aux}
    \leanok
    Let there be a set $P$ of points and a set $L$ of lines over a prime field, 
    with $|P| \leq n, |L| \leq n$ and $p^\beta \leq n \leq p^{2 - \beta}$,
    and each point contained in at least $n^{\frac12 - \stpfone(\beta)}$ lines
    and at most $4 n^{\frac12 + \stpfone(\beta)}$.
    Then the number of intersections is at most $C_3 n^{\frac32 - \stpftwo(\beta)}$.
\end{theorem}

\begin{proof}
    \leanok
    \uses{claim_342, ST_prime_field_proj}
    We use \ref{claim_342} to claim that there exist two points, $a, b$ such that for a large number of 
    points they are both on a line from $a$ and a line from $b$.
    We only keep those, and because all points are contained in $n^{\frac12 - \stpfone(\beta)}$ lines there are still many intersections.
    Then we remove all points on the line between $a$ and $b$. Because all lines, expect maybe one, intersect at most one such point,
    this step doesn't remove many intersections.
    Now we can apply \ref{ST_prime_field_proj}.
\end{proof}

\begin{theorem}
    \label{ST_prime_field_proj}
    \lean{ST_prime_field_proj}
    \leanok
    Let there be a set $P$ of points and a set $L$ of lines over a prime field, 
    with $|P| \leq n, |L| \leq n$ and $p^\beta \leq n \leq p^{2 - \beta}$,
    and each point contained in at least $n^{\frac12 - \stpfone(\beta)}$ lines
    and at most $4 n^{\frac12 + \stpfone(\beta)}$. Additionally, let there be two different points,
    $a, b$, such that not points in $P$ are on the line between $a$ and $b$, but all of them are one a line from $L$
    passing through $a$ and on a line from $L$ passing through $b$.
    Then the number of intersections is at most $C' n^{\frac32 - \sgone(\beta)}$.
\end{theorem}

\begin{proof}
    \leanok
    \uses{projective_transform, ST_grid}
    We use \ref{projective_transform} to apply a projective transformation to our lines and points.
    We want to argue that it moves the points to a grid, and preserves incidences, and then use
    \ref{ST_grid}.
\end{proof}